
\documentclass[12pt,a4paper]{mcmthesis}
\usepackage{ctex}
\usepackage{lipsum}
\usepackage{graphicx}
\usepackage{booktabs,colortbl}
\usepackage{xcolor}
\usepackage{tikz}
\usepackage{indentfirst}
\mcmsetup{CTeX = true,
        tcn ={\xiaowuhao 1001 }(\textcolor{red}{\textit{需要修改为自己队伍的实际队号}}), problem = A,
        sheet = true, titleinsheet = false, keywordsinsheet = true,
        titlepage = true, abstract = true}
\usepackage{newtxtext}
\usepackage{lipsum}


\usepackage{paralist}
\let\itemize\compactitem
\let\enditemize\endcompactitem
\let\enumerate\compactenum
\let\endenumerate\endcompactenum
\let\description\compactdesc
\let\enddescription\endcompactdesc

\setlength\abovedisplayskip{5pt}
\setlength\belowdisplayskip{-8pt}
\setlength{\parskip}{0.1em}

\newcommand\wordc[1]{\textbf{#1}}
\renewcommand{\appendixtocname}{附\quad录}
\renewcommand{\appendices}{\hspace{-2em}{\sanhao\HEI {\bf 附~~~录}}}
\colorlet{tableheadcolor}{gray!25} % Table header colour = 25% gray
\newcommand{\headcol}{\rowcolor{tableheadcolor}}

\title{\textcolor{red}{数学建模竞赛论文的题目()}}
\date{}

\usepackage{zhnumber} % change section number to chinese
\renewcommand\thesection{\zhnum{section}、\hspace{-1em}}
\renewcommand\thesubsection{\arabic{section}.\arabic{subsection}}

\usepackage[T1]{fontenc}
\usepackage[utf8]{inputenc}
\usepackage[font=small,labelfont={bf,sf},tableposition=top]{caption}

\makeatletter
   \renewcommand{\thefigure}{\ifnum \c@section>\z@ \arabic{section}-\fi \@arabic\c@figure}
   \renewcommand{\thetable}{\ifnum \c@section>\z@ \arabic{section}-\fi \@arabic\c@table}
\makeatother
\begin{document}

一.问题分析
为了更好解决高峰出行的问题,必须先了解高峰现象发生的实质,以及错峰方案的出行机制。

1.高峰的发生:排除突发事件的影响,高峰问题本质上是由于现有的地面交通基本条件限制例如行车道数量、路宽等使得人们的出行需求无法被完全满足。因此集中的上下班时间使得道路产生拥堵

2.实质分析:错峰本质是对乘客进行时间和空间上的分流,空间上的分流指的是不同线路的选择,这里假定居民出行线路选择偏好保持不变,仅考虑时间上的分流,即调节通勤通学的固定式时间。将高峰拆解成一个一个较小峰从而达到平峰的目的。

3.相关信息:通过结合线上线下办公的新常态的背景,以及疫情下对乘车人口密度的控制情况对已有信息进行建模,求出最佳错峰方案。由于有些城市已经开始实行错峰出行政策,将结果与现有政策进行一定程度上的对比分析。

二.模型建立:

1.信息概述
可以用(起始节点,目的节点)表示一条出行轨迹,放入新集合中。为简化计算可以将终节点到达时间视作上班时间,中间经过此时问题变为在已知起点、终点节点坐标以及路程中间经过的所有路段条件下,调节到达时间,使得各个节点流量每个时间控制在相对稳定的情况下。

2.基本假设

(1)乘客坐车时会选择最先来的车辆,且采用最短的乘车路径。

(2)终节点到达时间近似为上班时间。

(3)假设同一时间到达终节点的乘客上班时间可以统一调整。

3.公式表达

$x_{a}^{pq}$:乘客x的起始乘车点为p点,终止点是q,到达时间为a

\begin{eqnarray}
 I_{tu}(x_{a}^{pq}) & = & \left\{\begin{matrix}
  0& t时刻乘客x不在地铁u上 \\
  1& t时刻乘客x在地铁u上
\end{matrix}\right.
\end{eqnarray}

可通过列车的进出站点以及进出站时间,以及乘客刷卡乘车进出站节点与时间求出每个x对应的不同他t,u下示性函数的值。

此时目标函数变为
\begin{align}
  \min_{t,u}\max_{i} \sum_{x} I_{t}(x_{a}^{pq})
\end{align}

该模型的限制条件在于对a的调整,a每次调整的幅度$\Delta _{t}$为发车间隔,由于此时为高峰时间段,所以$\Delta _{t}$均为6分钟。本文建议采用贪心算法把所有合理可调整的变化大小进行遍历,最终可求得各个点最优解。

三.结果分析

文中仅展示早高峰错峰优化对应结果,且为初步粗略估计结果
\end{document}

