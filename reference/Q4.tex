\documentclass[12pt,a4paper]{mcmthesis}
\usepackage{ctex}
\usepackage{lipsum}
\usepackage{graphicx}
\usepackage{booktabs,colortbl}
\usepackage{xcolor}
\usepackage{tikz}
\usepackage{indentfirst}
\begin{document}

第四问分析
在充分考虑呼和浩特市城市快速路的基础之上,你能否提出一个地铁和公交互补的若干新增公交线路,以满足更多人员高峰出行时期的出行效率

1.问题分析:

在地面路况监控的过程中,我们发现在距离城区较远的地方也会存在拥堵情况,此时建造地铁成本较大,因此考虑利用灵活的公交线路对地铁未抵达区域进行互补。由于城市快速路主要位于呼和浩特市的中部偏东北部,因此考虑在西南部地区安排公交线路。

2.建造原则

在确定公交线路是需要遵循如下步骤:

STEP1:确定带规划区域大致范围,标出主要出行线路和道路等级。

STEP2:计算周围居民每天出行量与公交乘车出行意愿,在此基础上确定公交为环线或双向单线。

STEP3:找到最优最大覆盖公交路径,极大化公交居民出行意愿。


此外还要注意如下细节:

(1)线路长度适中,停靠站点数量合理

(2)做好与主干线的衔接,提供更多的换乘可能性

(3)结合周边环境、自然进行规划设计

3.总结反思

在实际情况中,公交选线一般将提案与实际情况相结合,按照经验与试运营情况权衡出合适的线路。关注城市的发展,与客流数据的动态变化,找到合适的方案。
\end{document}

