
\documentclass[12pt,a4paper]{mcmthesis}
\usepackage{ctex}
\usepackage{lipsum}
\usepackage{graphicx}
\usepackage{booktabs,colortbl}
\usepackage{xcolor}
\usepackage{tikz}
\usepackage{indentfirst}
\mcmsetup{CTeX = true,
        tcn ={\xiaowuhao 202105135237 }, problem = A,
        sheet = true, titleinsheet = false, keywordsinsheet = true,
        titlepage = true, abstract = true}
\usepackage{newtxtext}
\usepackage{lipsum}
\usepackage{cite}
\usepackage{amsmath}
\usepackage{paralist}
\let\itemize\compactitem
\let\enditemize\endcompactitem
\let\enumerate\compactenum
\let\endenumerate\endcompactenum
\let\description\compactdesc
\let\enddescription\endcompactdesc

\setlength\abovedisplayskip{5pt}
\setlength\belowdisplayskip{-8pt}
\setlength{\parskip}{0.1em}

\newcommand\wordc[1]{\textbf{#1}}
\renewcommand{\appendixtocname}{附\quad录}
\renewcommand{\appendices}{\hspace{-2em}{\sanhao\HEI {\bf 附~~~录}}}
\colorlet{tableheadcolor}{gray!25} % Table header colour = 25% gray
\newcommand{\headcol}{\rowcolor{tableheadcolor}}

\title{\textcolor{red}{数学建模竞赛论文的题目(三号黑体)}}
\date{}

\usepackage{zhnumber} % change section number to chinese
\renewcommand\thesection{\zhnum{section}、\hspace{-1em}}
\renewcommand\thesubsection{\arabic{section}.\arabic{subsection}}

\usepackage[T1]{fontenc}
\usepackage[utf8]{inputenc}
\usepackage[font=small,labelfont={bf,sf},tableposition=top]{caption}

\makeatletter
   \renewcommand{\thefigure}{\ifnum \c@section>\z@ \arabic{section}-\fi \@arabic\c@figure}
   \renewcommand{\thetable}{\ifnum \c@section>\z@ \arabic{section}-\fi \@arabic\c@table}
\makeatother

\begin{document}
\begin{abstract}
%abstract---------------
    {\song\xiaosihao
\setlength{\parindent}{2em}{轨道交通影响一个城市的运转效率。呼和浩特市在2019年建成地铁并投入使用,然而地铁的建设和维护需要消耗大量的财力和物力,需要考虑车厢节数、发车间隔、新站点选址这些指标,使其达到成本尽可能小,而乘客体验更优。所以我们就从这几个变量入手,进行优化模型的建立。}


\setlength{\parindent}{2em}{在第一个问题中,我们首先使用混合高斯模型对地铁客流量数据进行仿真,通过Fisher聚类方法对地铁运营的高峰、低峰时段进行划分。通过构建目标函数,加以条件限制得到4-6节车厢编组的建议,并给出了详细的发车时间间隔表。}

\setlength{\parindent}{2em}第二问,依据该市热点图和拥堵路段,确定需要改进的路段。采用最小二乘估计进行建模,之后对找到重点节点,对重点区域进行加权最小二乘估计的建模,拟合出所需的路线。对于地铁盈利的确定则是考虑影响成本的主要条件,加以限制,用优化模型解决。

\setlength{\parindent}{2em}第三问基于高峰情况发生的背后原因以及错峰出行的实质意义,建立优化模型,利用算法求得结果,输出出行时间与目的节点对应表格。为疫情防控提出切实可行的方案。

\setlength{\parindent}{2em}第四问在综合考虑公交地铁互补出行以及现有快速路的分布下,给出了公交线路规划方案以及一些注意事项。

\setlength{\parindent}{2em}最后对于模型的部分闲置进行总结和反思,总结了局限性和展望。
}


\begin{keywords}
{\song\xiaosihao
\textcolor{red}{使用到的模型名称、方法名称、特别是亮点一定要在关键字里出现,3$\sim$ 5个较合适。}}
\end{keywords}

\begin{itemize}
  \item \textcolor{blue}{前面一页必须使用模板格式(黑色部分),否则论文检测不通过。}
  \item \textcolor{blue}{此页为论文开始处,论文正文用阿拉伯数字从“1”开始连续编号,页码位于每页页脚中部。(目录可加可不加)}
\end{itemize}

\end{abstract}
\maketitle
\renewcommand{\contentsname}{\centerline{\sanhao\bfseries\HEI 目\quad 录}}
%\thispagestyle{empty}
%{\song\xiaosihao
\tableofcontents
%}

\newpage
\setcounter{page}{1}
\section{问题重述}
\subsection{引言}
%Introduction---------------

\setlength{\parindent}{2em}
{呼和浩特市地处中国的华北地区、北部地区,建成区面积约为260平方千米,常住人口在344万左右。截止2020年10月,呼和浩特市已经开通运营路线2条,地铁里程总长约49千米,车站总计43站。}

{在工作日,地铁通常用于满足人们通勤的需求,这样符合地铁建设的初衷,缓解交通高峰时段的交通拥堵。呼和浩特市的城市总面积较大,但常住人口相对较少,人口基数小,会直接应该影响地铁的当日客流量,从而直接影响运营的成本。}

{本文的目的就是想从车厢数量和发车间隔的确定入手,继而进行新增地铁线路的选址,并尽可能保证地铁可以营利。最后要基于地铁和公交现状,提出新的地铁和公交互补的公交线路,普惠于呼和浩特人民。}


\subsection{要解决的具体问题}

   1.当前的地铁运营是否合理?
   
   \begin{table}
  	\centering
  	\begin{tabular}{|l|l|l|}
  		\hline
  		& 早高峰 & 晚高峰 \\ \hline
  		工作日 & 7:00-9:00 & 17:00-19:00 \\ \hline
  		节假日 & 9:00-11:00 & 16:00-18:00 \\ \hline
  	\end{tabular}
  	\caption{早晚高峰时间表} 
  	\label{tab:早晚高峰}
  \end{table}
  呼和浩特市地铁运营时间为6:00-22:00,地铁型号为6B,每一列地铁最多容纳6*400人次。目前实行错峰发车,具体信息如 \ref{tab:早晚高峰}所示。在高峰发车间隔6分钟,在平峰时刻发车间隔10分钟,在晚20点以后发车间隔为12分钟。尽管在官方报道的客流量数据中,地铁客流量在逐渐增加,2020年12月6日,地铁客运量达到了17万人次,但是每天单向会发出100次,粗略的平均每列地铁的人次在400左右,载客率过低,也导致亏损十分严重。同时发车间隔也是影响成本的重要指标,在高峰时期的发车间隔短,可以优化乘客的乘车体验。在平峰和低峰时期,减少发车的次数,尽管会在一定程度上影响乘客的出行,但更大程度上会减少运营成本。所以我们需要先对当下的运营情况进行评估,然后给出最优车厢数量和发车间隔

  
  {2.建设新的线路选址在哪里?}
  
  {在新建新的线路的过程中,我们需要选择新的地点,来确保更多人会选择地铁这种运行方式,并且需要预测出每天至少需要多少人,才能满足地铁盈利的需求。}
 
  {3.实现学生党和工作党的错峰出行,实现平峰目标该怎么做?}
  
   {考虑到疫情的特殊背景,我们需要对不同站点的学生和工作者进行出行时间设定,以满足疫情防控的需要}
  
  {4.如何对公交新增线路,实现地铁公交更好的互补?}
  
   {将公交和地铁结合,通过相互补充满足更多人员高峰出行的需求。}
  
  {一些都可以用的东西。轨道交通影响整个城市的运转效率,发车间隔对整个轨道系统影响至关重要。运营公司自然希望在满足客流需求的太欧剑侠获得最大的收益,尽可能提升列车周转量与满载率。}
  {假设到站乘客服从均匀分布}



\section{名词解释与符号说明}

\subsection{名词解释与说明}
\begin{enumerate}
	\item \wordc{断面客流量:}在单位时间内,沿同一方向通过对到交通线路某断面你的乘客数量,即通过该断面你所在区间的客流量,分为上行断面客流量。
	
	\item \wordc{最大断面客流量:}在单位时间内,通过轨道交通线路各断面的客流量的最大值。
	
	\item \wordc{满载率:}反应车辆乘客满载程度的相对值,衡量车量利用程度的指标,可通过实际载客量与额定载客量之比求得。
	
	\item \wordc{乘客到达率:}乘客在某单位时刻在某一站点下车人数,单位通常为人/min。
	

	
	
\end{enumerate}
\subsection{主要符号与说明}

%tab1
\begin{table}[h!]
	\centering
	\small
	\begin{tabular}{p{60pt}<{\centering}|p{60pt}<{\centering}p{180pt}<{\raggedright}}
		\hline
		\headcol 序号 & 符号 & 符号说明 \\
		\hline
		1 & $M$ & 地铁车站的总数量 \\
		2 & $N$ & 当日的发车次数\\
		3 & $t_{i}^{j}$ & 第i辆地铁到达第j个站点的时间\\
		4 & $\Delta {t_i^j}$ & $t_{i-1}^{j}$到$t_i^j$的时间间隔\\
		5 & $F(\Delta {t_i^j})$ & 在$\Delta t_i^j$时间间隔内的进站人数\\
		6 & $f_j(t)$ & 站点j的乘客到达率函数 \\
		7 & ${f(t)}$ & 地铁线路的乘客到达率\\
		8 & $\Delta(t)$ & 发车间隔 \\
		9 & $\overline{F}$ & 地铁的最大荷载人数 \\
		10 & $\eta_i$ & 第i量车辆的载客率\\
		11 & $W_i$ & 乘客要乘第i量车的等待时间 \\
		12 & $K$ & k取值为1-43,一号线二号线依次编号,1a为1,1t为20,2a为1... \\
		13 & $\Omega$ & 所有站点指标的集合,如1a、2u。 \\
		$\cdots$ & $\cdots$\\
		\hline
	\end{tabular}
	%\caption{符号与说明}
	\label{symbol}
\end{table}



\section{问题分析}

本次比赛的四个问题,本质上也都是在围绕一个核心的意义-如何让地铁运营方的收益和乘客的体验感同时可以到达一个更好的水平,也就是如何更好的“以乘客为本,兼顾效益”。

从乘客的角度,影响体验感的变量会有:平均等待时长、票价、地铁空间体验感等等。从运营者角度需要考虑的问题就是:发车间隔、站点选址、票价定价、运营成本。


\subsection{数据分析}
\begin{figure}[h!t]
	\centerline{\includegraphics[scale=0.4]{发车时间图}\quad
	}
	\caption{\song\wuhao
		发车时间图}
	\label{fig:发车时间图}
\end{figure}

附件一给出了每一个地铁站的经纬度坐标,容易发现,一号线共有20个站点,二号线共有24个站点,二号线的交点为新华广场站,在一号线中标号为$1h$,在二号线中标记为$2k$,地铁一号线的总运行时间在45分钟,二号线的总运行时间在47分钟。同时根据当前的发车时间间隔数据,可以求出发车时间表,如下图\ref{fig:发车时间图}

附件二中给出了在9月1日到9月14日共计14天的,自早上6点到晚上22点45分的各站点进出站的数据。数据间隔为15分钟,以9月1日6点的1A为例,进站数据为62人,即在6点到6点15期间有62人刷卡进站。由于具体时点的人员分布对于解决问题来讲没有那么重要,所以在每一个15分钟我们都认为,乘客的到达服从均匀分布,在上例中,就是每分钟约有4人进站。同时我们也认为乘客的等待时间服从均匀分布,如果是在非高峰期,那么就是服从均值为5,取值在[0,10]的均匀分布,如果在高峰期就是均值为3,取值为[0,6]的均匀分布。

此处的附件二给出了某一时点的进站数据和出站数据,对于进站的人员来说,既可以在两个地铁方向中选择一个方向(如果是中转站,就是4个方向),又可以在众多站点中选择一个终点,由于目前我们的数据无法获得,但此处依旧想引入一个转移矩阵的概念,前面定义了K,并且用数字代替了站名,用1代表1a,此处我们用$\Omega$代表所有点,其中$\Omega={1a,1b,1c...,2v,2w,2x}$表示所有站点的集合。我们设转移矩阵为X,其中$X_{ij}$表示从i个站点出发,到j个站点下车的概率。比如说,$X_{1a1k}$是指从伊利健康谷上车,从艺术学院下车的概率。这个概率的计算需要用到当日各闸机数据,通过整理所有路线的人次,用$\frac{单线程客流量}{总线路客流量}$即可获得一个转移概率矩阵,这一矩阵可以更明显的看出热点线路和热点区域。从而可以对某一站点的进站人员的大概率经过的行程进行估计。而本题中所给的数据,我们简单认为,进站人数中,以0.5的概率向两个方向行进。


\subsection{地铁客流随时间的分布研究}
\begin{figure}[h!t]
	\centerline{\includegraphics[scale=0.4]{客流图}\quad
	}
	\caption{\song\wuhao
		客流量图}
	\label{fig:客流量}
\end{figure}


我们将附件2的数据进行简要的处理,通过求和求出每个时间节点的总进站人数与总出站人数,得到了如下所示的图\ref{fig:客流量}。

在图中,橙色的线表示的是出站的数据,蓝色的是进站的数据,在发车的前一个半小时,我们进站数据比出站数据大,这是合情合理的,但是随着时间的推移,到14:15点附近时,会发现橙色的线与蓝色的线会形成一个空白部分,这部分在一定程度上代表着有大量的人滞留在地铁站内,这与实际情况是不太相符合的,因为地铁全程是45分钟左右,那么滞留旅客在1-2小时内均会出站完毕,参考其他城市的进站出站客流量图,在运行4个多小时后进出站数据会在一定程度上重合,而并非附件二的数据这样,上午会有大量的人不出站,而下午和晚上出站的人数又远远多于进站。同时通过所给出的数据进行分析,我们采取了9月1日到9月6日的数据进行描点,得到的图片中,如下图\ref{fig:客流量}所示,仔细观察不难看出,6天的数据边缘重合度较高,每一时点和下一个时点的变化量之间没有什么明显的变化,无论是平日还是节假日,都是有一个进站的高峰和一个出站的高峰。在这种现实情况下,乌鲁木齐地铁并不需要将工作日与节假日的高峰期与平峰期等分开。


在图\ref{fig:客流量}中,一天之内会形成两个较为平缓的客流高峰,分别位于8:15左右和16:00左右。并且早高峰的峰值会更高一些,说明早高峰的客流比较集中。晚高峰的客流相对缓和。工作日与非工作日的波动性差距不大,非工作日的数据会工作日的数据更上移一些。

\begin{figure}[h!t]
	\centerline{\includegraphics[scale=0.4]{客流图}\quad
	}
	\caption{\song\wuhao
		客流量图}
	\label{fig:客流量}
\end{figure}

\subsection{乘客体验感指标}
平均等待时间:在前面我们已经陈述,在此处我们认为等待时间服从均匀分布,所以此时的平均等待时间的期望为3分钟,5分钟,和6分钟。

乘客舒适度体验:地铁车厢的拥挤程度会直接影响乘客的乘车舒适度,我们会沿用车辆内乘客站立人员。在此处因为呼和浩特的地铁人员偏稀少,所以我们不对这部分加以展开,就选择一个合适的密度来进行计算。在此处,我们在求车厢节数的时候,极限站席密度会用$7/m^{-2}$.这种情况下的拥挤程度为感到有些拥挤,站席范围有些突破。

\subsection{问题一的分析}

 通过前面对于数据的初步观察,可以看出目前的运营模式下有几点不太贴合现实情况,所以我们初步可以认定目前发车方案的不合理。理由包括但不限于
 
 1.地铁空载率过高,在每一个时刻,以一号线为例,一列地铁运行时间为45分钟,那么这个过程中,以平峰发车的间隔十分钟为例,每个时刻,在整个一号线双向行驶中,至少有8列地铁,可以承载$8*400*6$人次,而在每一时刻,最大的断面流量\textcolor{red}{xxxx人次},远远少于
 
 
 
 {问题一首先需要对数据进行简单地处理,对于通过数据可以得出的结论进行总结和讨论。然后通过数据来推测最优车厢节数,以及最优发车间隔。最优车厢节数的推算,需要用到最大断面流量。最大断面流量的获得先需要对客流特征进行提取,建立客流到达率函数。通过到到达率函数用Fisher有序样本聚类分析,得到客流量高峰、低峰和平峰,用于错峰发车,划分调动时段。最后考虑乘客的等车时间成本和车辆的利用率用遗传算法确定发车间隔。再通过混合高斯模型进行数据的仿真,对模型进行检测。}


\subsection{问题二的分析}
问题二需要根据呼和浩特的城市特点,提出合理的选址方案,让民众更加倾向于选择地铁这种出行方式,这样才能更大程度上缓解交通的拥堵情况。在这一部分,我们是以事实为依据,对于现实生活中实时路况信息进行调查,对于拥堵严重的地区,考虑用地铁缓解交通的压力。
确定热点位置之后,通过最小二乘法确定路线。

{问题二同样包含一个地铁运营的盈利边界问题。这里会涉及到地铁运营成本模型,考虑的现金的时间价值问题。}
{中国的城市轨道发展至今,地铁运营行业一直需要面对的课题就是:如何提高地铁运营企业的经济效益,如何降低运营成本,如何增加客票收入,实现“减亏”。
	
通常意义下定义的地铁运营企业的收支平衡,是指剔除了还本付息、折扣和大修基金之后的相对平衡,也就是在考虑收支平衡的时候不考虑前期的成本。在(地铁成本运营分析中)有写地铁的完全成本,包括建设成本,营业支出(包括地铁在运营期内发生的运营成本)其他三项支出(折旧、计提大修基金和利息)。

地铁的前期建造成本以亿为单位进行计算,如果考虑完全成本,那么地铁盈利是基本不可能的,所以我们也同样不考虑建设成本,只计算电费、维修、人工、管理和相关税费,与票价冲抵之后的平衡。}

 {首先对未来的客流量进行一定的预测。由于我们可收集到的数据有限,只有6组单日客流量数据以及题目中给出的仿真数据,官方数据如下所示,所以我们此处只引入集中计算未来客流量的简单模型。}
\begin{table}
    \centering
    \begin{tabular}{|l|l|}
    \hline
        日期 & 单日客流量 \\ \hline
        2019/12/29 & 58171 \\ \hline
        2019/12/30 & 57714 \\ \hline
        2019/12/31 & 82211 \\ \hline
        2020/1/1 & 119613 \\ \hline
        2020/10/31 & 162629 \\ \hline
        2020/12/5 & 166234 \\ \hline
        2020/12/6 & 171687 \\ \hline
    \end{tabular}
\end{table}
    由于呼和浩特市地铁正在蓬勃发展,那么我们可以假设增速为一定常数来进行建模。在实际应用时,考虑$Y$ 时需要将工作日数据与节假日数据分开进行预测。具体公式如下所示
    \begin{equation}
        \begin{aligned}
            Y_{计}=Y{基} \times (1+\beta)^t
            \end{aligned}
    \end{equation}
    $Y_{计}$为计划年度运量,$Y_{基}$为基期年度运量,$\beta$ 为年平均增长率,t为年数,然后会根据这个数据进行预测到达多少人才会实现地铁运营的目标。




\section{模型假设}

\subsection{通用假设}
\subsection{问题一假设}
\begin{enumerate}
	\item 假设所给数据能推出的结论是合理的。
\end{enumerate}
\subsection{问题二假设}
\begin{enumerate}
	\item 网站的路面信息可以反映真实的交通拥堵状况。
	\item 假设各区域人口服从均匀分布,在本题不考虑居民的分布情况。
	\item 在地铁选址的过程中,不考虑实际的地点是否可行。
	\item 在考虑地铁的运营成本时,不考虑地铁的前期成本。

\end{enumerate}

\subsection{问题三假设}
\subsection{问题四假设}

\section{模型的建立与求解}

\subsection{问题一的分析和求解}

\subsubsection{行车间隔优化模型的建立}

\subsubsection{行车间隔优化模型的求解}

\subsubsection{结果分析}

\subsection{问题二的分析和求解}
由于题目中只要求我们选择地址,但是因为同时存在多个热点区域,所以为了缓解地面交通压力,降低碳排放,现拟建地铁三号线。优化地铁规划建设是建设新地铁线路前必须解决的问题。因为科学的规划建造方案,才能形成有层次性、稳定性的地铁线网结构。
\subsubsection{限制条件的推导}
\begin{figure}[h!t]
	\centerline{\includegraphics[scale=0.6]{呼市地图}\quad
	}
	\caption{\song\wuhao
		呼市地图}
	\label{fig:呼市地图}
\end{figure}
首先对该地区进行总体分析,下图 \ref{fig:呼市地图}为该地区地图,通过对地图上主要交通路线分布情况呼和浩特主要交通路线聚集在城市中部以及西北方向。
\begin{figure}[h!t]
	\centerline{\includegraphics[scale=0.6]{城市热力图}\quad
	}
	\caption{\song\wuhao
		城市热力图}
	\label{fig:城市热力图}
\end{figure}
如下 \ref{fig:城市热力图}为城市热力图,可以在一定程度上影响该市的交通状况,颜色越深的地方越有可能人员聚集,距离中心较远区域地广人稀,且自然山脉地形限制,暂不考虑将地铁延拓到这类相关区域。市区人员出行人流、车流量最大一般在市中心,副中心,居民区,工业区等。
\begin{figure}[h!t]
	\centerline{\includegraphics[scale=1]{拥堵图}\quad
	}
	\caption{\song\wuhao
		拥堵图}
	\label{fig:拥堵图}
\end{figure}


通过实时监测路况拥堵程度,提取地面车辆密度信息。如下图 \ref{fig:拥堵图},是以兴茂家园附近路线为例,其中绿色到黄色再到红色分别表明畅通到拥挤不同程度的路况。由于本文旨在缓解交通压力,减少碳排放,因此地面车辆密度的高低可以很好的反应相关信息。通过对路况信息的监控,拥堵区域集中在小区、学校、大型市场附近。

综上所述,我们可以大致给出关于地铁三号线的建设的初步考虑:1.穿过城市中心。2.为缓解地面交通压力,要在尽可能多的途径拥堵路段或其附近区域。3.由于建造维护成本高,暂不考虑覆盖偏远地区。

\subsubsection{模型建立}
1.结构基础
在三线地铁的基本结构中,三角形结构最为稳定,根据《成都地铁规划》中,换乘次数、换乘压力、线网吸引力覆盖强度,达到最优。在此基础上我们寻找需要覆盖的重要节点。

2.	引入重要节点
在确定了线路整体趋势的基础上,地铁规划又一重要因素在于停车站点的设置。设有k个节点,用集合S表示。为有效控制建设成本,我们设置k<20,超过时将次要点淘汰。
为此我们按照重要性优先级顺序先后考虑如下三类节点:

\begin{figure}[h!t]
	\centerline{\includegraphics[scale=1]{经纬度图}\quad
	}
	\caption{\song\wuhao
		经纬度图}
	\label{fig:经纬度图}
\end{figure}
I :地面拥堵点
监测地面实时路况时,仅保留主要日常交通要道监测对象排除了拥堵的立交桥,快速内环,国道,省道的情况。在此基础上排除附近1km内有地铁的路段(假定通过一些措施可以对地面交通进行分流,提高地铁出行率)。通过观察24小时拥堵情况变化,利用Python提取出城市中心区域17个高拥堵点的经度纬度。如 \ref{fig:经纬度图}所示:
对于给定的关键大交通流量节点,以及学校工作聚集地,利用requests函数获取关键节点的经纬度信息。基于现有节点信息

II :换乘站点
用于和1、2号线衔接,建造更高覆盖率、更方便快捷的地铁网。站点选择参考进出站热度,以及利用编程软件计算出的待分流区域到各个站点三维坐标轴下距离平方和,为参考依据,最后选择1j和2i作为3号线与1/2号线的换乘站点。

III :过渡/其他站点
若两个站点之间距离较远,可以选择中间新增一个站点,满足更多乘客的出行需求。若地铁线附近存在有标志性建筑(商圈、公园等),可以考虑增加相关站点。此站点的选择可以根据初步得到的线路进行进一步优化,目前暂时无法确定。


\begin{figure}[h!t]
	\centerline{\includegraphics[scale=1]{选址散点图}\quad
	}
	\caption{\song\wuhao
		选址散点图}
	\label{fig:选址散点图}
\end{figure}

3.模糊规划
为简化计算,文中通过经纬度信息近似距离信息,假设在平面直角坐标系下,该信息依旧有较高的参考价值。其中蓝色代表现有1/2号线对应站点红色代表待分流拥堵点 \ref{fig:选址散点图}。
(1)直接拟合
从图中可以看到标出的点(出去左上角的点)呈现明显的圆环趋势,因此考虑直接拟合环形地铁,但这类地铁制造、维护成本较高,在接下来尝试是否有更加接近直线的地铁建造方案。
(2)线性规划
在考虑近似直线地铁建造规划时,首先想到利用最小二乘法进行线性回归。这种方法可以拟合出一条直线到各散点的距离差平方和最小。
但是这种方法的弊端在于使得各个点所含的信息量相同,与实际情况出入较大。我们应该考虑尽可能地接近人流量大的拟合点,为了优化现有的线性规划模型,我们考虑在进行最小二乘法距离求和时,给每个点的距离平方引入权重系数。
权重的确定采用打分系统,80%由交通拥堵程度M决定,20%由周边建筑信息N决定。
M的确定:在监控24小时路况时,针对每小时拥堵情况进行累加。
\begin{table}
	\centering
	\begin{tabular}{|l|l|l|l|}
		\hline
		路况 & 畅通(绿色) & 较拥挤(黄色) & 拥挤(红色) \\ \hline
		计数 & 0 & 1 & 1.5 \\ \hline
	\end{tabular}
\end{table}
\begin{equation}
	\begin{aligned}
		W_i=0.8*M_i/(sum(M_i))+0.2*N_i/(sum(N_i))
	\end{aligned}
\end{equation}
4.实际规划
上部分线路的确定仅是提供模糊规划方案,实际线路确定应当围绕所给直线结合实际地形、地质等多方面因素综合考虑。由于信息有限,这里仅提供建议。
(1)	互补原则:和现有机动车快速干道没有长距离重复
(2)	低成本性:避免大量拆迁建筑
等等

\subsection{问题 三的求解和分析 的求解和分析 的求解和分析}

\subsubsection{对问题的分析}

问题 三要求我们 $\cdots$。

\subsubsection{对问题的求解}

\section{模型的评价与推广 模型的评价与推广}

\textcolor{red}{将模型进行数值计算,并与附件中的真实采样值(进行列表或图示)比较。对误差进行数据分析,给出误差分析的理论估计。}

\subsection{模型的评价}


1. 优点

\textcolor{red}{得到满意的解、
较好地解决了$\cdots$问题、
使模型得到简化、
使结果更合理,避免…带来的较大误差、
使问题描述比较清晰、
减少大的计算量。}



\begin{figure}[h!t]
\centerline{\includegraphics[scale=0.8]{fig4.pdf}}
\caption{\song\wuhao 图~3的标题名称}
\end{figure}


(4)运用多种数学软件(如 MATLAB、SPSS),取长补短,使计算结果更加),取长补短,使计算结果更
加 准确、明晰.

2. 缺点

\textcolor{red}{主观性过强、
建立在什么的前提条件下、
有一定的局限性、
存在不确定性、
有一定的偏差。
}
.

\subsection{模型的、模型的 推广}

\begin{itemize}

\item \textcolor{red}{对本文中的模型给出比较客观的评价,必须实事求是,有根据,以便评卷人参考。}

\item \textcolor{red}{推广和优化,需要花费功夫想出合理的、甚至可以合理改变题目给出的条件的、不一定可行但是具有一定想象空间的准理想的方法、模型。由此做出一些改进方向,也可以是参赛者一些来不及实现的思路。}
\end{itemize}

.

\section{模型的改进}

\subsection{模型一的改进}


\subsection{模型二的改进}
三.反思与不足
1.实际情况反思
在现有地铁经过的站点附近1km为半径的环形区域内,路面交通情况依旧不容乐观,存在着较多时段的拥堵,人们在选择出行方式时,更偏好于地面交通。因此在可以考虑进行低碳出行宣传,以及制定相关政策,对地铁出行方式的偏斜。以便更好地缓解交通压力和碳排放。

2.在相关数据的采集方面可以更加完善,在站点设置上可以通过调查问卷方式了解当地居民的看法,合理化地铁规划。同时可以对不同区域进行细分,基于更多的信息考虑各个区域的权重。


