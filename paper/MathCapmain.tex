
\documentclass[12pt,a4paper]{mcmthesis}
\usepackage{ctex}
\usepackage{lipsum}
\usepackage{graphicx}
\usepackage{booktabs,colortbl}
\usepackage{xcolor}
\usepackage{tikz}
\usepackage{indentfirst}
\mcmsetup{CTeX = true,
        tcn ={\xiaowuhao 202105135237 }, problem = A,
        sheet = true, titleinsheet = false, keywordsinsheet = true,
        titlepage = true, abstract = true}
\usepackage{newtxtext}
\usepackage{lipsum}
\usepackage{cite}
\usepackage{amsmath}
\usepackage{paralist}
\newcommand{\itemize}{\compactitem}
\newcommand{\enditemize}{\endcompactitem}
\newcommand{\enumerate}{\compactenum}
\newcommand{\endenumerate}{\endcompactenum}
\newcommand{\description}{\compactdesc}
\newcommand{\enddescription}{\endcompactdesc}

\setlength\abovedisplayskip{5pt}
\setlength\belowdisplayskip{-8pt}
\setlength{\parskip}{0.1em}

\newcommand\wordc[1]{\textbf{#1}}
\renewcommand{\appendixtocname}{附\quad录}
\renewcommand{\appendices}{\hspace{-2em}{\sanhao\HEI {\bf 附~~~录}}}
\colorlet{tableheadcolor}{gray!25} % Table header colour = 25% gray
\newcommand{\headcol}{\rowcolor{tableheadcolor}}

\title{\textcolor{red}{数学建模竞赛论文的题目(三号黑体)}}
\date{}

\usepackage{zhnumber} % change section number to chinese
\renewcommand\thesection{\zhnum{section}、\hspace{-1em}}
\renewcommand\thesubsection{\arabic{section}.\arabic{subsection}}

\usepackage[T1]{fontenc}
\usepackage[utf8]{inputenc}
\usepackage[font=small,labelfont={bf,sf},tableposition=top]{caption}

\makeatletter
   \renewcommand{\thefigure}{\ifnum \c@section>\z@ \arabic{section}-\fi \@arabic\c@figure}
   \renewcommand{\thetable}{\ifnum \c@section>\z@ \arabic{section}-\fi \@arabic\c@table}
\makeatother

\begin{document}
\begin{abstract}
%abstract---------------
    {\song\xiaosihao
\setlength{\parindent}{2em}{轨道交通影响一个城市的运转效率。呼和浩特市在2019年建成地铁并投入使用,然而地铁的建设和维护需要消耗大量的财力和物力,需要考虑车厢节数、发车间隔、新站点选址这些指标,使其达到成本尽可能小,而乘客体验更优。所以我们就从这几个变量入手,进行优化模型的建立。}


\setlength{\parindent}{2em}{在第一个问题中,我们首先使用混合高斯模型对地铁客流量数据进行仿真,通过Fisher聚类方法对地铁运营的高峰、低峰时段进行划分。通过构建目标函数,加以条件限制得到4-6节车厢编组的建议,并给出了详细的发车时间间隔表。}

\setlength{\parindent}{2em}第二问,依据该市热点图和拥堵路段,确定需要改进的路段。采用最小二乘估计进行建模,之后对找到重点节点,对重点区域进行加权最小二乘估计的建模,拟合出所需的路线。对于地铁盈利的确定则是考虑影响成本的主要条件,加以限制,用优化模型解决。

\setlength{\parindent}{2em}第三问基于高峰情况发生的背后原因以及错峰出行的实质意义,建立优化模型,利用算法求得结果,输出出行时间与目的节点对应表格。为疫情防控提出切实可行的方案。

\setlength{\parindent}{2em}第四问在综合考虑公交地铁互补出行以及现有快速路的分布下,给出了公交线路规划方案以及一些注意事项。

\setlength{\parindent}{2em}最后对于模型的部分闲置进行总结和反思,总结了局限性和展望。
}


\begin{keywords}
{\song\xiaosihao
\textcolor{red}{使用到的模型名称、方法名称、特别是亮点一定要在关键字里出现,3$\sim$ 5个较合适。}}
\end{keywords}

\begin{itemize}
  \item \textcolor{blue}{前面一页必须使用模板格式(黑色部分),否则论文检测不通过。}
  \item \textcolor{blue}{此页为论文开始处,论文正文用阿拉伯数字从“1”开始连续编号,页码位于每页页脚中部。(目录可加可不加)}
\end{itemize}

\end{abstract}
\maketitle
\renewcommand{\contentsname}{\centerline{\sanhao\bfseries\HEI 目\quad 录}}
%\thispagestyle{empty}
%{\song\xiaosihao
\tableofcontents
%}

\newpage
\setcounter{page}{1}
\section{问题重述}
\subsection{引言}
%Introduction---------------

\setlength{\parindent}{2em}
{呼和浩特市地处中国的华北地区、北部地区,建成区面积约为260平方千米,常住人口在344万左右。截止2020年10月,呼和浩特市已经开通运营路线2条,地铁里程总长约49千米,车站总计43站。}

{在工作日,地铁通常用于满足人们通勤的需求,这样符合地铁建设的初衷,缓解交通高峰时段的交通拥堵。呼和浩特市的城市总面积较大,但常住人口相对较少,人口基数小,会直接应该影响地铁的当日客流量,从而直接影响运营的成本。}

{本文的目的就是想从车厢数量和发车间隔的确定入手,继而进行新增地铁线路的选址,并尽可能保证地铁可以营利。最后要基于地铁和公交现状,提出新的地铁和公交互补的公交线路,普惠于呼和浩特人民。}


\subsection{要解决的具体问题}
\begin{enumerate}
  \item {由于呼市的流动人口较少,选择地铁的人数较少,目前地铁的总容纳量为2400人次,但实际上每天的最多时刻的人流量为---人次,那么有很多空间和电力等资源被浪费,那么我们需要根据给定的模拟进出站数据,对地铁上的人数进行预测,从而确定在资源可以合理配置下的最优车厢数量。因为对于地铁来讲,不同时刻的人流量波动较大,什么样的发车间隔同样也很重要。合适的发车间隔会提升用户使用感。}
  \item {在新建新的线路的过程中,我们需要选择新的地点,来确保更多人会选择地铁这种运行方式,并且需要预测出每天至少需要多少人,才能满足地铁盈利的需求。}
  \item {考虑到疫情的特殊背景,我们需要对不同站点的学生和工作者进行出行时间设定,以满足疫情防控的需要}
  \item {将公交和地铁结合,通过相互补充满足更多人员高峰出行的需求。}

  {一些都可以用的东西。轨道交通影响整个城市的运转效率,发车间隔对整个轨道系统影响至关重要。运营公司自然希望在满足客流需求的太欧剑侠获得最大的收益,尽可能提升列车周转量与满载率。}
  {假设到站乘客服从均匀分布}
\end{enumerate}

\section{问题分析}

\textcolor{red}{我们将在这一部分对所给数据的应用进行阐述,解释部分变量存在的实际意义。对于问题给出近似可用的模型解释。}

\subsection{问题一的分析}

 \textcolor{red}{对问题1研究的意义的分析。
问题1属于$\cdots\cdots$数学问题,对于解决此类问题一般数学方法的分析。
对附件中所给数据特点的分析。
对问题1所要求的结果进行分析。
由于以上原因,我们可以将首先建立一个$\cdots\cdots$的数学模型I,然后将建立一个$\cdots\cdots$的模型II,$\cdots\cdots$对结果分别进行预测,并将结果进行比较.
}


\subsection{问题二的分析}
 \textcolor{red}{对问题2研究的意义的分析。
问题2属于$\cdots\cdots$数学问题,对于解决此类问题一般数学方法的分析。
对附件中所给数据特点的分析。
对问题2所要求的结果进行分析。
由于以上原因,我们可以将首先建立一个$\cdots\cdots$的数学模型I,然后将建立一个$\cdots\cdots$的模型II,$\cdots\cdots$对结果分别进行预测,并将结果进行比较.
}

\subsection{附件分析}
{附件一给出了地铁站点经纬度信息,在此基础上我们需要根据确定的站点信息,给出地铁在各站点的运行时间间隔,以备使用。呼和浩特一号线一共有20个站点,二号线共有24个站点。通过相关数据查阅,地铁一号线的总运行时间在45分钟,二号线的总运行时间在47分钟。}

{附件二中给出了在9月1日到9月14日共计14天的,自早上6点到晚上22点45分的各站点进出站的数据。数据间隔为15分钟,以9月1日6点的1A为例,进站数据为62人,即在6点到6点15期间有62人刷卡进站。}

{对于刷卡进站的人,有两个方向可以选择,并且可能在其中的任何一站下车,所以我们更多关注下车乘客这一变量。}

\subsection{地铁客流随时间的分布研究}
{人们的生活会影响地铁客流的变化,同时人流量随着时间的分布同样可以为我们所用。如\ref{fig:客流量}所示,一天之内会形成两个较为平缓的客流高峰,分别位于8:15左右和16:00左右。并且早高峰的峰值会更高一些,说明早高峰的客流比较集中。晚高峰的客流相对缓和。工作日与非工作日的波动性差距不大,差距在于非工作日的出行人数会偏多。(地铁线路通常会有通学通勤的特性)}

\begin{figure}[h!t]
	\centerline{\includegraphics[scale=0.4]{客流图}\quad
	}
	\caption{\song\wuhao
		客流量图}
	\label{fig:客流量}
\end{figure}


\section{模型假设}
\begin{enumerate}
  \item 模型的假设要结合整个模型的建立作出的一个合理的假设,不能过于理想化,要尽量切合实际问题的处理来做出相应的合理的假设;
  {模型一的假设中,有需要进行车厢节数的预测的,根据《城市轨道交通工程项目建设标准》建议的车内乘客站立人员密度评价标准。我们设定的我们的站席密度为7,也就是人口最为拥挤的情况也不能超过每平米7人的限制。}
  \item 模型假设二;
  \item 模型假设三;
\end{enumerate}

\section{名词解释与符号说明}

\textcolor{red}{一般都会有符号解释和说明,对于一些装有的专有名词解释,需要的时候就需要对其进行解释与说明,我们以下面几个例子为例。}

\subsection{名词解释与说明}
\begin{enumerate}
\item \wordc{理论通行能力:}理论通行能力是指每一条车道~(或每一条道路) 在单位时间内
能够通过的最大交通量。

\begin{figure}[h!t]
\centerline{\includegraphics[scale=0.4]{客流图}\quad
}
\vspace{-4em}
\caption{\song\wuhao
图~1的标题名称}
\label{fig:4-1}
\end{figure}

关于如\ref{fig:4-1}所示插图、绘图、表格以及公式等相关资源请点击~\href{http://www.latexstudio.net}{\textcolor{blue}{\LaTeX{}工作室}}

\item \wordc{修正通行能力:}在具体条件下,通过修正系数对理论通行能力修正后得到的单
位时间内所能通过的最大交通量。

\end{enumerate}
\subsection{主要符号与说明}

%tab1
\begin{table}[h!]
  \centering
  \small
  \begin{tabular}{p{60pt}<{\centering}|p{60pt}<{\centering}p{180pt}<{\raggedright}}
   \hline
   \headcol 序号 & 符号 & 符号说明 \\
   \hline
    1 & $\nu$ & 行车速度(km/h) \\
    2 & t$_{\min}$ & 车头最小时距(s) \\
    3 & $J_{\rm a}$ & 车头最小间隔(m) \\
    4 & $J_{\rm z}$ & 车辆平均长度(m) \\
    5 & $J_{\gamma}$ & 车辆的制动距离(m) \\
    6 & $J_{\max}$ & 司机在反应时间内车辆行驶的距离(m) \\
    7 & $A_{\max}$ & 最大交通量 \\
    8 & $\alpha_{1}$ & 车道数修正系数 \\
    9 & $\alpha_{2}$ & 车道宽度和侧向净宽修正系数 \\
    10 & $\alpha_{3}$ & 大型车修正系数 \\
    11 & $\alpha_{4}$ & 驾驶员技术水平修正系数 \\
    12 & $K_{j}$ & 阻塞密度 \\
    13 & $\nu_{f}$ & 自由车速 \\
    $\cdots$ & $\cdots$\\
    \hline
  \end{tabular}
  %\caption{符号与说明}
  \label{symbol}
\end{table}

\section{模型的建立与求解}

数据的预处理:
1. $\cdots\cdots$数据全部缺失,不予考虑。
2. 对数据测试的特点,如周期等进行分析。
3. $\cdots\cdots$数据残缺,根据数据挖掘等理论根据$\cdots\cdots$变化趋势进行补充。
4. 对数据特点(后面将会用到的特征)进行提取。
   用$\cdots\cdots$软件聚类分析和各个不同问题的需要,采得$\cdots\cdots$组采样,每组5-8个采样值。将采样所对应的特征值进行列表或图示。
根据数据特点,对总体和个体的特点进行比较,以表格或图示方式显示。


\subsection{问题一的分析和求解}

首先,我们分析并提炼附件2中客流量的数据特征,使用高斯混合模型 (GMM, Gaussian mixture model) 结合随机噪声 (random noise) ,生成了5组仿真模拟数据。随后,根据地铁的发车间隔和运行速度,我们编写代码模拟 (simulate) 了整个地铁的运行过程。最后,我们统计了每趟列车的车载人数的变化情况,分析得出最优的车厢数量方案,并使用聚类算法 (cluster algorithm) 分段求解最优的发车间隔。

\subsubsection{时间度量的映射}

我们设 $t$ 是一个表示时间的变量。$t$ 的值加1表示过了一分钟。我们设定当 $t=0$ 时表示一天的开始时刻,则 $t=720$ 时表示一天的正午时刻, $t=1440$ 时表示一天的结束时刻。地铁的首班列车发车时间是早上6时,此时 $t=360$,末班列车发车时间是晚上22时,此时 $t=1320$。

我们考虑一个连续型的概率密度函数 (Probability Density Function),不妨记为 $P(x)$。由于 $P(x)$ 是一个定义在 $(-\infty,\infty)$ 上的函数,而地铁一天的运行时间只有16个小时,因此我们需要对时间建立一个从 $(-\infty,\infty)$ 到 $[360,1320]$ 的映射。设映射后的概率密度函数为 $Q(t)$,映射算子为 $T(\cdot)$,映射过程如公式\ref{eq:p1_def-map-continuous}所示。映射过程的含义是,仅考虑$P(x)$ 在 $[360,1320]$ 上的取值,再做一个归一化操作。

\begin{equation}
    Q(t) = T(P(t)) = \frac{1}{\int_{360}^{1320}P(x)dx} P(t)
    \label{eq:p1_def-map-continuous}
\end{equation}

其中,$\int_{360}^{1320}P(x)dx$ 是归一化因子,确保概率密度函数 $Q(t)$ 在区间 $[360,1320]$ 上的积分值为1。

同理,对于离散型的概率密度函数 $Q(s)$,设其定义域为集合 $S$。映射算子 $T(\cdot)$ 的定义如公式\ref{eq:p1_def-map-disperse}所示。映射过程的含义是,仅考虑$P(x)$ 在集合 $S$ 上的取值,再做一个归一化操作。

\begin{equation}
    Q(s) = T(P(s)) = \frac{1}{\sum_{x \in S} x} P(s)
    \label{eq:p1_def-map-disperse}
\end{equation}

\subsubsection{高斯混合模型的建立}

高斯混合模型 (GMM, Gaussian mixture model) 是多个高斯分布 (Normal Distribution) 的叠加。形式化地,一个均值为 $\mu$ 方差为 $\sigma^2$ 的高斯分布,其概率密度函数 $P(x;\mu,\sigma^2)$ 如公式\ref{eq:p1_normal-distribution}所示。

\begin{equation}
    P(x;\mu,\sigma^2) = \frac{1}{\sqrt{2\pi}\sigma} \exp \left( - \frac{(x-\mu)^2}{2\sigma^2} \right)
    \label{eq:p1_normal-distribution}
\end{equation}

现有 $N$ 个高斯分布,其均值分别为 $\mu_1$、$\mu_2$、$\cdots$、$\mu_N$,其方差分别为 $\sigma^2_1$、$\sigma^2_2$、$\cdots$、$\sigma^2_N$,则其概率密度函数分别为 $P_1(x;\mu_1,\sigma^2_1)$、$P_2(x;\mu_2,\sigma^2_2)$、$\cdots$、$P_N(x;\mu_N,\sigma^2_N)$。设每个高斯分布发生的概率为 $\alpha_1$、$\alpha_2$、$\cdots$、$\alpha_N$,则高斯混合模型 $P_0(x)$ 可按照公式\ref{eq:p1_def-gmm-continuous}构建。

\begin{equation}
    P_0(x) = \sum_{i=1}^{N} \alpha_i P_i(x;\mu_i,\sigma^2_i)
    \label{eq:p1_def-gmm-continuous}
\end{equation}

其中,$\sum_{i=1}^{N} \alpha_i=1$。

\subsubsection{客流量数据的特征分析}

附件2中的客流量数据存在诸多明显不符合实际情况的地方,其中两处最明显的错误是:

\begin{itemize}
  \item 每天早上出站人数均远多于进站人数。例如对于9月1日的数据,截至早上9时,累计进站11517人次,累计出站14591人次,这在实际生活中是明显不可能出现的情况。
  \item 末班车在晚上22时由起点站发车,但是在末班车开走之后,仍有旅客进站。例如对于9月1日的数据,1号线于22时由1a站和1t站分别发出当天的最后一班列车,但是在22:15分还有8人进入1a站,在22:30分还有9人进入1a站,在22:15分还有10人进入1a站,这些旅客不可能再坐上任何一趟地铁。
\end{itemize}

由于附件2中数据的明显错误,我们不可能使用附件2中的数据。取而代之,我们分析了附件2中数据的特点,并基于这些数据特征重新生成了若干组仿真模拟数据。并在生成的模拟数据上进行建模研究。

我们在分析了附件2中数据的基础上,概括提炼出了如下的数据特征:

\begin{itemize}
  \item 每日客流约为10万人次,其中9月10日客流量最少为77984人次,9月7日客流量最多,为122495人次。
  \item 出站高峰期在早上9时左右,出站客流量大约为2000人次每15分钟。
  \item 进站高峰期在晚上18时左右,进站客流量大约为2500人次每15分钟。
  \item 早上9时,1a-1e、2u站的进站客流量较大,大约在50-100人次每15分钟每站。
  \item 晚上18时,1e、1f、1m、1p、1s、1t、2b-2i、2k、2q、2u站的进站客流量较大,大约在50-200人次每15分钟每站。
  \item 早上9时,1m-1p、1s-1t、2b、2e-2k、2u站的出站客流量较大,大约在50-200人次每15分钟每站。
  \item 晚上18时,1a-1b、1e-1f、1m、1p、2u站的出站客流量较大,大约在50-150人次每15分钟每站。
  \item 从早上10时到晚上17时,1f、1m、1p、2q、2u站的出站客流量持续较大,大约在50-150人次每15分钟每站。
\end{itemize}

\subsubsection{客流量数据的生成模型}

我们定义二维离散型的概率密度函数 $Q_{in}(t,s)$ 表示一名旅客在时刻 $t$ 从 $s$ 站进站的概率,同理用 $Q_{out}(t,s)$ 表示一名旅客在时刻 $t$ 从 $s$ 站进站的概率。其中 $t$ 的取值范围是区间 $[360,1320]$ 中的整数,$s$ 的取值范围为集合
\[ S = \left\{ 1a,1b,\cdots,1t,2a,2b,\cdots,2x \right\} \]

我们考虑 $Q_{in}(t,s)$ 和 $Q_{out}(t,s)$ 的边缘概率密度函数 (Marginal Probability Density Function)

\begin{equation*}
    R_{in}(t) = \sum_{s \in S} Q_{in}(t,s) \\
    R_{out}(t) = \sum_{s \in S} Q_{out}(t,s)
\end{equation*}

$R_{in}(t)$ 和 $R_{out}(t)$ 反映了在 $t$ 时刻所有站点的进站概率和出站概率,是服从混合高斯分布的概率密度函数,其形式化定义如公式\ref{eq:p1_def-r}所示。

\begin{equation}
    R_{in}(t) = T \left( \max \left( \sum_i \alpha_i P_i(t;\mu_i,\sigma^2_i), c \right) \right) \\
    R_{out}(t) = T \left( \max \left( \sum_i \alpha_i P_i(t;\mu_i,\sigma^2_i), c \right) \right)
    \label{eq:p1_def-r}
\end{equation}

其中,$c$ 是一个介于0到1之间的小数,定义了每个时刻的最小概率。这样做的目的是,防止高斯分布的“$3-\sigma$原则”导致某些时刻的概率近乎于0。

在 $t_0$ 时刻,$Q_{in}(t_0,s)$ 和 $Q_{out}(t_0,s)$ 反映了在 $t_0$ 时刻站点 $s$ 的进站概率和出站概率,也是服从混合高斯分布的概率密度函数,其形式化定义如公式\ref{eq:p1_def-q}所示。

\begin{equation}
    Q_{in}(t_0,s) = T \left( \max \left( \sum_i \alpha_i P_i(s;\mu_i,\sigma^2_i), c \right) \right) \\
    Q_{out}(t_0,s) = T \left( \max \left( \sum_i \alpha_i P_i(s;\mu_i,\sigma^2_i), c \right) \right)
    \label{eq:p1_def-q}
\end{equation}

\subsubsection{客流量数据的生成示例}

\begin{table}[h]
    \centering
    \caption{每时刻进出站概率的高斯混合模型}
    \label{tab:p1_gmm-time}
    \begin{tabular}{c|c|cc|cc}
        \hline
        \multirow{2}{*}{时刻} & \multirow{2}{*}{$t$} & \multicolumn{2}{c|}{进站 GMM} & \multicolumn{2}{c}{出站 GMM} \\ \cline{3-6}
                   &                        & 权重 & 高斯分布      & 权重 & 高斯分布      \\ \hline
                   &                        & 0.4 & $P(t;8,0.5)$ & 0.4 & $P(t;8,0.5)$ \\
        6时 - 22时 & $\left[360,1320\right]$ & 0.4 & $P(t;18,0.5)$ & 0.4 & $P(t;18,0.5)$ \\
                   &                        & 0.2 & $P(t;21,0.2)$ & 0.2 & $P(t;21,0.2)$ \\ \hline
    \end{tabular}
\end{table}


\subsubsection{***模型的建立}

模型建立的内容要点如下:

模型的主要类别:

几种常见的建模目的:

建模过程常见的几个要点:

模型的基本要求:

模型选择要点:

加分项(能在规定时间内做完后还有足够时间的再考虑加分项):

1、鼓励创新。在能解决问题的基础上,对经典模型进行改进,欣赏独树一帜、有创新性的模型,但要合理。

2、对于同一问题使用两个或以上合理模型进行求解。避免出现单纯罗列模型,又不做对比和评价的现象。

\begin{figure}[h!t]
\centerline{
\begin{tikzpicture}[scale=0.6]
  \foreach \y [count=\n] in {
      {74,25,39,20,3,3,3,3,3},
      {25,53,31,17,7,7,2,3,2},
      {39,31,37,24,3,3,3,3,3},
      {20,17,24,37,2,2,6,5,5},
      {3,7,3,2,12,1,0,0,0},
      {3,7,3,2,1,36,0,0,0},
      {3,2,3,6,0,0,45,1,1},
      {3,3,3,5,0,0,1,23,1},
      {3,2,3,5,0,0,1,1,78},
    } {
      \foreach \x [count=\m] in \y {
        \node[fill=yellow!\x!blue, minimum size=6mm, text=white] at (\m,-\n) {\x};
      }
    }
\end{tikzpicture}\quad
\begin{tikzpicture}[scale=0.6]
  \foreach \y [count=\n] in {
      {74,25,39,20,3,3,3,3,3},
      {25,53,31,17,7,7,2,3,2},
      {39,31,37,24,3,3,3,3,3},
      {20,17,24,37,2,2,6,5,5},
      {3,7,3,2,12,1,0,0,0},
      {3,7,3,2,1,36,0,0,0},
      {3,2,3,6,0,0,45,1,1},
      {3,3,3,5,0,0,1,23,1},
      {3,2,3,5,0,0,1,1,78},
    } {
      \foreach \x [count=\m] in \y {
        \node[fill=yellow!\x!purple, minimum size=6mm, text=white] at (\m,-\n) {\x};
      }
    }
\end{tikzpicture}
}
\caption{图~2的标题名称}
\end{figure}



参考话术:我们需要解决的问题是$\cdots\cdots$,题目要求是$\cdots\cdots$,剔除$\cdots\cdots$数据后选用何种类型的模型优点进行分析。具体步骤123$\cdots$



\subsubsection{***模型的求解}

\textcolor{red}{将预处理数据带入上述模型,通过$\cdots$软件得到$\cdots$结果。(编程代码详见附件*)。模型求解及结果需要图文并茂,用数据说话  用图展示。具体步骤123$\cdots$}
\begin{align}
A_{\max}& =\dfrac{3600}{t_{\min}}=\dfrac{3600}{J_{\min} /(v / 3.6)}
=\dfrac{1000 v}{J_{\min }}(\text{1 } / h) \\
J_{\min}& =J_{\rm r}+J_{z}+J_{\rm a}
\end{align}


\subsubsection{***结果}

\textcolor{red}{针对于每一个问题的结果综述总结。}



\subsection{问题 三的求解和分析 的求解和分析 的求解和分析}

\subsubsection{对问题的分析}

问题 三要求我们 $\cdots$。

\subsubsection{对问题的求解}

\textbf{模型 Ⅱ—基于 负荷度 负荷度 分析 的小区开放影响度综合评价}

(1)模型的准备

1)负荷度介绍

负荷度( V/CV/CV/C)是指在理想条件下,最大服务交通量与基本行能力之比.

2)数据处理

将道路分为主干和次,其要参数详见 表 10

\begin{table*}[h!]
  \centering
  \small
  \tabcolsep 2.5pt
  \caption{主次道路参数表}
\begin{tabular*}{0.8\linewidth}{p{60pt}<{\centering}p{60pt}<{\centering}
p{60pt}<{\centering}p{80pt}<{\centering}p{80pt}<{\centering}}
\toprule
  道路类型  &  主干路  &  支干路  &  小区内宽道路  &  小区内窄道路  \\
  \midrule
  行车速度  & 50 km / h & 40 km / h & 30 km / h & 20 km / h \\
 车道数  & 4 & 3 & 2 & 1 \\
\bottomrule
  \end{tabular*}
  \label{tab10}
\end{table*}

(2)模型的建立

1)小区的分类

根据小区结构,周边道路分布形状和周边道路车道数的不同,我们将小区分
别分为~4、2、3 类,小区的分类结果详见表~11


2)计算周边各路段及交叉口的通行能力



对于周边各路段的通行能力,我们运用问题二已建立的模型进行计算.在此
基础上对于交叉口的通行能力交叉口~G 我们建立公式如下:

\begin{align}
G_{\text{交又口}}& =\sum_{i=1}^{n} G_{i} \\
G_{i}& =\sum_{j=1}^{k} C_{j}
\end{align}


其中,$C_{j}$ 为进口各车道的通行能力,$ G_{i}$ 为交叉口各进口的通行能力.


3)建立影响度综合评价体系~[9][10][11]

我们采用先单项评价再综合评价的方法,其总体思路见表~12

\begin{table*}[h!]
  \centering
  \small
  \tabcolsep 2.5pt
  \caption{小区分类表}
\begin{tabular*}{0.8\linewidth}{p{100pt}<{\centering}|p{60pt}<{\raggedright}|p{180pt}<{\raggedright}}
\hline
分类标准 & 类型名称& 类型说明\\
\hline
\multirow{4}*{小区结构 }& A组团有序型 & 小区楼房呈组团型分布,每一区域间隔较大,开放后小区
道路较宽,且区域间分布有序\\

& B紧凑有序型 & 小区楼房间隔紧凑,且排列有序,开放后道路网格呈“街
区型”,特点为“高密度、窄路宽.\\
 &C组团无序型& 小区楼房呈组团式分布,每一区域间隔较大,开放后小区
道路较宽,但区域间分布杂乱小区楼房间隔紧凑,但排列杂乱,开放后小区道路呈现“低\\
&D紧凑无序型&密度,窄路宽”的特点\\

\multirow{2}*{周边道路形状分布}& 四周围绕型&四周均为道路\\

&半边包围型&半边围绕道路\\

\multirow{3}*{车道数(针对半封闭性)}& 主干道型 & 两条道路均为主干道\\

&次干道型 & 两条道路均为次干道\\

&混合型& 两条道路一主一次\\
\hline
  \end{tabular*}
  \label{tab11}
\end{table*}

\begin{table*}[h!]
  \centering
  \small
  \tabcolsep 2.5pt
  \caption{综合评价思路表}
\begin{tabular*}{0.8\linewidth}{p{100pt}<{\centering}|p{160pt}<{\raggedright}|p{80pt}<{\raggedright}}
\hline
 评价性质  &  评价内容  &  评价指标  \\
 \hline
\multirow{2}*{ 单项评价 } & \multirow{2}*{  局部路段及交叉口交通负荷影响 } &  路段影响度  \\
& &交叉口影响\\
\multirow{2}*{ 综合评价 } & \multirow{2}*{整个路网交通负荷影响} &平均路段影响度  \\
&&平均交叉口影响度\\
\hline
  \end{tabular*}
  \label{tab12}
\end{table*}

A. 负荷度单项评价

a. 封闭式小区开放后,新增小区内道路对于周边某一路段i 的影响度 $K_{si}$
根据公式计算:
\begin{align}
K_{s i}&=\dfrac{I_{s i p}-I_{s i b}}{B_{s i}} \\
I_{s i p}& =I_{s i b}+a
\end{align}

其中,$I _{sip}$ 为小区道路建成后路段 i 上高峰小时交通量,$I _{sib}$ 为不考虑小区道
 路建成后新增交通量的情况下,路段 i 的高峰小时交通量,  $B_{s i}$  为路段 $i$ 的设计
 通行能力,$a$ 为开放后小区道路的通行量.
b. 封闭式小区开放后,新增小区内道路对于周边道路交叉口的影响度  $K_{c i}$
根据公式计算:
\begin{align}
  K_{c i}=\dfrac{I_{c i p}-I_{c b}}{B_{c t}}
\end{align}


其中,$K_a$ 为小区道路建成后对交叉口 i 的影响度,$I_{crp}$ 为小区道路建成后交 叉口 $i$
上高峰小时交通量, $ I_{c i b}$  为不考虑小区道路建成后新增交通量的情况下, 交叉口 i 的
高峰小时交通量,  $B_{c i}$  为交叉口 $i$ 的设计通行能力.


\begin{figure}[h!t]
\centerline{\includegraphics[scale=1]{fig1.pdf}}
\caption{\song\wuhao 图~3的标题名称}
\end{figure}


\section{模型的评价与推广 模型的评价与推广}

\textcolor{red}{将模型进行数值计算,并与附件中的真实采样值(进行列表或图示)比较。对误差进行数据分析,给出误差分析的理论估计。}

\subsection{模型的评价}


1. 优点

\textcolor{red}{得到满意的解、
较好地解决了$\cdots$问题、
使模型得到简化、
使结果更合理,避免…带来的较大误差、
使问题描述比较清晰、
减少大的计算量。
}

(1)问题求解中 辅之流程图, 将建模思路完整清晰的展现出来;

(2)问题二在对 问题二在对理论通行能力进修复时考虑因素 细致、全面,理论通行能力进修复时考虑因素
 细致、全面,系数准确度高;

(3)在问题三中,提出“影响度”的概念较为直观地定量给小区开放后的效果,简便有.在影响度计算上由
点及面从每个路段、交叉口到整 个路网,层深入具有逻辑性;

\begin{figure}[h!t]
\centerline{\includegraphics[scale=1]{fig4.pdf}}
\caption{\song\wuhao 图~3的标题名称}
\end{figure}


(4)运用多种数学软件(如 MATLAB、SPSS),取长补短,使计算结果更加),取长补短,使计算结果更
加 准确、明晰.

2. 缺点

\textcolor{red}{主观性过强、
建立在什么的前提条件下、
有一定的局限性、
存在不确定性、
有一定的偏差。
}

(1)在数学软件的计算中会将小数计算 结果进行保留,使得随后的会将小数计算 结果进行保留,使得随后
的或统计结果造成一定误差;

(2)问题二求解修正通行能力时多次使用了查表,操作不够简便.

\subsection{模型的、模型的 推广}

\begin{itemize}

\item \textcolor{red}{对本文中的模型给出比较客观的评价,必须实事求是,有根据,以便评卷人参考。}

\item \textcolor{red}{推广和优化,需要花费功夫想出合理的、甚至可以合理改变题目给出的条件的、不一定可行但是具有一定想象空间的准理想的方法、模型。由此做出一些改进方向,也可以是参赛者一些来不及实现的思路。}
\end{itemize}

1. 问题二中 建立 的模型 在现实 生活 中可以 作为 检验 数据 对实测数据 的准确 性进行 检验,帮助 人们
更好 的测算 交通 数据.

2. 基于问题三建立的模型,可以根据道路实时检测数(某段单位间内 基于问题三建立的模型,推算新建
一条道路对于当前交 通状况的改善效果,帮助度等).

\section{模型的改进}

\subsection{模型一的改进}
针对问题二中的模型一,在具体求解大型车对车辆通行能力的修正系数时,
我们利用交通量的测算值对照得到相应的大型车修正系数.但是,在实际操作中
交通量的测定有很大的难度,如果此时交通量数据无法得到,那么我们便不能得
到相应的修正系数,因此我们对模型进行改进.

由~GREENSHIELD K-V 线性模型,可得通行能力的公式:
\begin{align}
A_{p}=\begin{cases}
\dfrac{3600}{t}\left(1-\dfrac{3.6 l}{V_{t} t}\right)\left(V_{f}>7.2 l / t\right) \\
\dfrac{250 V_{f}}{t}\left(V_{f} \leq 7.2 l / t\right)
\end{cases}
\end{align}

对应的临界车辆速度:
\begin{align}
V_{p}=\begin{cases}
\dfrac{V_{f}-3.6 l}{t} & \left(V_{f}>7.2 l / t\right) \\
\dfrac{1}{2} V_{f} & \left(V_{f} \leq 7.2 l / t\right)
\end{cases}
\end{align}

由美国道路通行能力准则可得,美国将道路服务水平分为六级:A-F 级,而
我国目前针对当前国情,将道路服务水平分成四级:一级相当于美国的A、B 两
级;二级相当于美国的C 级;三级相当于美国的D 级;四级相当于美国的E、F
级。因此,相应的,将美国服务水平划分标准进行针对性修正,得到中国道路服
务水平划分标准,见表

\begin{table*}[h!]
  \centering
  \small
  \tabcolsep 2pt
  \caption{我国服务水平划分标准}
\begin{tabular*}{0.87\linewidth}{p{60pt}<{\centering}p{40pt}<{\centering}
p{40pt}<{\centering}p{40pt}<{\centering}p{40pt}<{\centering}
p{80pt}<{\centering}p{40pt}<{\centering}}
\toprule
服务水平 (L0S)  & \multicolumn{2}{c} {一级 } & 二级  & 三级  & \multicolumn{2}{c} {四级 } \\
\cline{2-3}\cline{6-7}
服务交通量  & 800 & 1200 & 1800 & 2500 & $A_{D}$ & $\leqslant A_{P}$ \\
 速度  km / h & 120 & 120 & 120 & 120 & $\geqslant V_{p}$ & $\leqslant V_{p}$ \\
 V / C & 0.33 & 0.48 & 0.71 & 1.0 & $A_{p} / A_{\max}\leqslant 1.0$ & -(无意义 ) \\
\bottomrule
  \end{tabular*}
\end{table*}

由于车流量的测算相对于交通量来说较易得到,我们便可以不用对交通量进
行测算,可以通过车流量与通行能力的比值计算出~V/C 饱和度值,再通过该值对
照我国服务水平划分标准,间接得到服务交通量,从而得到大型车对通行能力的
修正系数.


\subsection{模型二的改进}

针对于问题三中的模型,在得出各个类型小区在开放后对于整个小区周边路
网交通负荷影响度后,无法判别小区开放的效果是积极的还是消极的,由此我们
可以采用~Bress 悖论的原理进行判别:在个人独立选择路径的情况下,为某路网
增加额外的通行能力(如增加路段的等),反而会导致整个路网的整体运行水平
降低的情况.

将路网进行简化如图~15:

根据推导可得: 当 $\beta_{3}/\left(\beta_{1}+\beta_{2}\right) \leq\left(\beta_{5}+\beta_{6}\right)/\beta_{4}$ 时,会发生悖论,即道路的开
通反而会加剧原有道路的交通状况.

\textcolor{red}{需重新起页,不得与论文正文内容在同一页上}

\begin{rmk}
5篇以上!
\end{rmk}

\newpage

\begin{thebibliography}{99}
\addcontentsline{toc}{section}{参考文献}
\bibitem{1} 李向鹏. 城市交通拥堵对策——封闭型小区交通开放研究~[D]. 交通运输工程,
2014.4.
\bibitem{2} 司守奎等. 数学建模算法与应用~[M]. 北京:国防工业出版社,2011.8 第一版;
\bibitem{3} 吕彬. 城市居住区“开放性”模式研究~[D]. 建筑设计,2006.6.
\bibitem{4} 茹红蕾. 城市道路通行能力的影响因素研究~[D]. 交通运输工程,2008.3.
\bibitem{5} VISSIM 软件路网搭建教程.
http://wenku. baidu.com/view/7bc33214680203d8ce2f24c4.html
\bibitem{6} 赵琳,邵长桥. 基于~VISSIM 的高速公路基本路段实际通行能力仿真分析~[J]. 道
路交通与安全,2007.2.
\bibitem{7} 李冬梅,李文权. 道路通行能力的计算方法 [J]. 河南大学学报,2002.6:24-27.
\bibitem{8} 城市轨道施工安全及交通组织 [S].2014.
\bibitem{9} 李鑫, 李雪等. 城市道路网络脆弱性评估指标研究综述~[J]. 公路交通科技,
2016.1:155-157.
\bibitem{10} 詹斌, 蔡瑞东等. 基于城市道路网络脆弱性的小区开放策略研究 [J]. 技术方法,
2016.7:98-101.
\bibitem{11} 彭驰. 物流园区交通影响分析研究~[D]. 交通运输工程,2007, 4.

\end{thebibliography}
\newpage

\begin{appendices}

\section*{}

\textbf{\textcolor[rgb]{0.98,0.00,0.00}{程序一:MATLAB算道路车辆通行能力:}}
\lstinputlisting[language=Matlab]{./code/mcmthesis-matlab1.m}

\section*{}

\textcolor[rgb]{0.98,0.00,0.00}{\textbf{程序二:C++ 求解路网正体影响度:}}
\lstinputlisting[language=C++]{./code/mcmthesis-sudoku.cpp}

\newpage
\def\thesection{A}
\renewcommand{\thetable}{\wuhao A-\arabic{table}}
\setcounter{table}{0}
\section*{数据表格}
\textcolor[rgb]{0.98,0.00,0.00}{\textbf{表格数据:}}
\input{Appendices1}

\end{appendices}
\end{document}
%%
%% This work consists of these files mcmthesis.dtx,
%%                                   figures/ and
%%                                   code/,
%% and the derived files             mcmthesis.cls,
%%                                   mcmthesis-demo.tex,
%%                                   README,
%%                                   LICENSE,
%%                                   mcmthesis.pdf and
%%                                   mcmthesis-demo.pdf.
%%
%% End of file `mcmthesis-demo.tex'.
