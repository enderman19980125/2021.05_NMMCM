%abstract---------------
    {\song\xiaosihao
\setlength{\parindent}{2em}{轨道交通影响一个城市的运转效率。呼和浩特市在2019年建成地铁并投入使用,然而地铁的建设和维护需要消耗大量的财力和物力,需要考虑车厢节数、发车间隔、新站点选址这些指标,使其达到成本尽可能小,而乘客体验更优。所以我们就从这几个变量入手,进行优化模型的建立。}


\setlength{\parindent}{2em}{在第一个问题中,我们首先使用混合高斯模型对地铁客流量数据进行仿真,通过Fisher聚类方法对地铁运营的高峰、低峰时段进行划分。通过构建目标函数,加以条件限制得到4-6节车厢编组的建议,并给出了详细的发车时间间隔表。}

\setlength{\parindent}{2em}第二问,依据该市热点图和拥堵路段,确定需要改进的路段。采用最小二乘估计进行建模,之后对找到重点节点,对重点区域进行加权最小二乘估计的建模,拟合出所需的路线。对于地铁盈利的确定则是考虑影响成本的主要条件,加以限制,用优化模型解决。

\setlength{\parindent}{2em}第三问基于高峰情况发生的背后原因以及错峰出行的实质意义,建立优化模型,利用算法求得结果,输出出行时间与目的节点对应表格。为疫情防控提出切实可行的方案。

\setlength{\parindent}{2em}第四问在综合考虑公交地铁互补出行以及现有快速路的分布下,给出了公交线路规划方案以及一些注意事项。

\setlength{\parindent}{2em}最后对于模型的部分闲置进行总结和反思,总结了局限性和展望。
}
